\documentclass[11pt,italian]{article}
\usepackage[T1]{fontenc}
\usepackage[utf8]{inputenc} %utf8 % lettere accentate da tastiera
\usepackage[italian]{babel} % lingua del documento
\usepackage{blindtext}
\usepackage{enumitem}
\usepackage{float}
\usepackage{xcolor}   % for \textcolor
\usepackage[font=small,labelfont=bf,skip=10pt]{caption}
\usepackage{subcaption}
\setlength{\belowcaptionskip}{5pt}
\usepackage{listings}
\lstset{
  basicstyle=\ttfamily,
  columns=fullflexible,
  frame=single,
  breaklines=true,
  postbreak=\mbox{\textcolor{red}{$\hookrightarrow$}\space},
}
\usepackage{hyperref}
\usepackage{cleveref}
\usepackage{graphicx}
\graphicspath{ {./images/} }

% Use lstinline as item in description
\makeatletter
\newcommand*{\lstitem}[1][]{%
  \setbox0\hbox\bgroup
    \patchcmd{\lst@InlineM}{\@empty}{\@empty\egroup\item[\usebox0]\leavevmode\ignorespaces}{}{}%
    \lstinline[#1]%
}
\makeatother

\title{Multiple Sequence Alignment (MSA) \\ di sequenze SARS-CoV-2}

\date{A.A.: 2019/2020}

\author{
    \textsc{Edoardo Silva} 816560 \\
    \textsc{Davide Marchetti} 815990
}

\begin{document}
\maketitle

\section{Abstract}
Per ogni variazione registrata nell\'output della parte1, viene creata un entità nella tabella di output contenente:

\begin{itemize}
    \item \textbf{gene\_id}: id del gene in cui cade la variazione
    \item \textbf{gene\_start}: inizio del gene in cui cade la variazione
    \item \textbf{gene\_end}: fine del gene in cui cade la variazione
    \item \textbf{cds\_start}: inizio della \textit{coding DNA sequence} della porzione del gene in cui cade la variazione
    \item \textbf{cds\_end}: fine della \textit{coding DNA sequence} della porzione del gene in cui cade la variazione
    \item \textbf{original\_codone}: codone della reference prima della modifica
    \item \textbf{altered\_codone}: codone della reference modificati dalla variazione
    \item \textbf{relative\_start}: inizio della variazione in rispetto all'inizio della cds
    \item \textbf{relative\_end}: fine della variazione in rispetto all'inizio della cds
    \item \textbf{sequence}: variazione
    \item \textbf{encoded\_aminoacid}: amminoacido codificato da \textbf{altered\_codone}
\end{itemize}
% aminoacids_lookup_table
\section{codice}
	Il codice inizia caricando la sequenza di reference dal file \textbf{'reference.fasta\'} e un file di output.\newline % qui il codice può cambiar e%
	Infine carica in una lista le CDS della reference dal file \textbf{'Genes-CDS.xlsx\'}. \newline
	Il  corpo del codice esegue un ciclo per ogni variazione nel file di output: \newline
	Nel ciclo è inserito un controllo per la sequenza CDS da unire: prendo valori dalla tabella e se trovo valori nella collonna \textbf{'join\_at'}, eseguo sulla nuova CDS. 
	\begin{lstlisting}[basicstyle=\small\ttfamily,caption=Porzione di ciclo,label=code:variations_to_genes = []]
{
for key, value in variations:
        ...
        
        %#affected_cds = df_ref_cds.loc[(value['from'] > df_ref_cds['from']) & (value['to'] < df_ref_cds['to'])]
	for index, cds in affected_cdses.iterrows():
            cds_sequence = reference[cds['from']:cds['to'] + 1]
            if pd.notna(cds['join_at']):
                join_at = int(cds['join_at'])
                cds_sequence = reference[cds['from']:join_at + 1] + reference[join_at:cds['to'] + 1]
        %#if not affected_cds.empty:
            gene = df_ref_genes.loc[affected_cds['GeneID']]
            gene_id = affected_cds.iloc[0]['GeneID']
            gene_start = gene.iloc[0]['Start']
            gene_end = gene.iloc[0]['End']
            cds_start = affected_cds.iloc[0]['from']
            cds_end = affected_cds.iloc[0]['to']
            sequence = value['alt']
            relative_start = value['from'] - gene_start
            relative_end = relative_start + len(sequence),
    ...
  }
}
\end{lstlisting}
	e riempie la tebella
\begin{lstlisting}[basicstyle=\small\ttfamily,caption=Appending nel file di output,label=code:variations_to_genes]
{
variations_to_genes.append({
	'gene_id': gene_id,
	'gene_start': gene_start,
	'gene_end': gene_end,
	'cds_start': cds_start,
	'original_codone': original_codone,
	'altered_codone': altered_codone,
	'relative_start': relative_start + 1,
	'relative_end': relative_end,
	'sequence': sequence,
	'original_aminoacid': original_aminoacid,
	'encoded_aminoacid': encoded_aminoacid
  })
}
\end{lstlisting}	
	da generare come file output.
	\newpage
\begin{lstlisting}[basicstyle=\small\ttfamily,caption=Tabella per la traduzione in amminoacidi,label=code:aminoacids_table]
{
aminoacids_lookup_table = {
    "START" : 'ATG',
    "STOP" : ["TAA", "TAG", "TGA"],
    'F' : ['TTT', 'TTC'],
    'L' : ['TTA', 'TTG', 'CTT', 'CTA', 'CTC', 'CTG'],
    'I' : ['ATT', 'ATC', 'ATA'],
    'M' : ['ATG'],
    'V' : ['GTT', 'GTA', 'GTC', 'GTG'],
    'S' : ['TCT', 'TCA', 'TCC', 'TCG', 'AGT', 'AGC'],
    'P' : ['CCT', 'CCA', 'CCC', 'CCG'],
    'T' : ['ACT', 'ACA', 'ACC', 'ACG'],
    'A' : ['GCT', 'GCA', 'GCC', 'GCG'],
    'Y' : ['TAT', 'TAC'],
    'H' : ['CAT', 'CAC'],
    'Q' : ['CAA', 'CAG'],
    'N' : ['AAT', 'AAC'],
    'K' : ['AAA', 'AAG'],
    'D' : ['GAT', 'GAC'],
    'E' : ['GAA', 'GAG'],
    'C' : ['TGT', 'TGC'],
    'W' : ['TGG'],
    'R' : ['CGT', 'CGA', 'CGC', 'CGG', 'AGA', 'AGG'],
    'G' : ['GGT', 'GGA', 'GGC', 'GGG']
}
}
\end{lstlisting}	
	
\section{output}
\begin{lstlisting}[basicstyle=\small\ttfamily,caption=Generazione di output,label=code:output]
	columns = ['gene_id', 'gene_start', 'gene_end', 'cds_start', 'cds_end', 'original_codone', 'altered_codone', 'relative_start', 'relative_end', 'sequence', 'original_aminoacid', 'encoded_aminoacid']
	df_variations_to_genes = pd.DataFrame(variations_to_genes, columns=columns)
	df_variations_to_genes.to_csv(os.path.join('..', 'output', 'out.csv'))
\end{lstlisting}	
Genera file \textbf{out.csv} nella sotocartella \' output \' .
% inserire immagine output?%

\section{conclusione}
Eccetto per la variazione con cancellazione di sequenze, tutti gli altri codoni modificati sono traducibili.

\end{document}