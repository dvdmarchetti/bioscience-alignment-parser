\documentclass[12pt]{article}
\usepackage{blindtext}
\usepackage{imakeidx}
\usepackage[utf8]{inputenc}
\usepackage[T1]{fontenc}
\usepackage{enumitem}

\title{Multiple Sequence Alignment (MSA) di sequenze SARS-Cov2}

\date{21/05/2020}
\author{Silva Edoardo 816560, Marchetti Davide 815990}

\begin{document}
\maketitle
%\makeindex

\section*{Abstract}

	Usando i sequenziamente genomici del virus denominato Covid-19, reperibili nul sito della banca genetica https://www.ncbi.nlm.nih.gov/genbank/sars-cov-2-seqs/ e https://www.gisaid.org/, abbiamo allineato un insieme di sequenze relativi a paesi mediorientali:
	\begin{itemize}
		\item Iran: MT320891.2, MT281530.2, hCov-19/Iran/KHGRC-1.1-IPI-8206/2020|EPI\_ISL\_442523|2020-03-09, hCoV-19/Iran/HGRC-2-2162/2020|EPI\_ISL\_437512|2020-03-26. %identificativi%.
		\item Israele: MT276598.1, MT276597.1, hCoV-19/Israel/130710062/2020|EPI\_ISL\_447469|2020-04-14.
		\item Pakistan: MT262993.1, MT240479.1, hCoV-19/Pakistan/Gilgit1/2020|EPI\_ISL\_417444|2020-03-04.
		\item Turchia: MT327745.1, hCoV-19/Turkey/HSGM-10232/2020|EPI\_ISL\_437334|2020-03-24, hCoV-19/Turkey/HSGM-1027/2020|EPI\_ISL\_437317|2020-03-27.
	\end{itemize}
	E allineati grazie ai tools Clustal Omega e MUSCLE , reperibili al sito https://www.ebi.ac.uk/Tools/msa/; al fine di ottenere le differenze con la sequenza di riferimento ottenuta su un campione di Whuan NC\_045512.2.\newline
	\textbf{Il lavoro consiste in identificare le differenze tra le sequenze e tra gli allineamenti dei 2 tool sulle medesime sequenze}. \newpage
\section{Descrizione}
	
	L'analisi filogenetica serve a ricostruire la storia delle mutazioni, ossia come specie antiche si sono evolute in quelle moderne.\newline
	I geni sono composti da sequenze di Acido Desossiribonucleico (DNA) (o RNA in caso di alcuni virus), composto da uno zucchero(dessossiribosio) che unisce un gruppo fosfato ad una base azotata per comporre nucleotidi (i nucleotidi sono unità ripetitive costitutive degli acidi nucleici) che compongono il genoma.\newline
	Un GENE non è nient' altro che una particolare sequenza di DNA, che codifica l' informazione in un linguaggio a quattro lettere, nel quale ogni lettera è rappresentata da una base.\newline
	Il genotipo di un individuo è dato dal suo corredo genetico.
Il fenotipo, invece, è l'insieme dei caratteri che l'individuo manifesta:dipende dal suo genotipo, dalle interazioni fra geni e anche da fattori
esterni.\newline 
\newline Il DNA viene utilizzato: \begin{enumerate} \item trascrivendolo in RNA	
\item rendere l' RNA stabile aggiungendo ai limiti 7-metilguanosina (7mGTP)
\item trascritto il risultato precedente in amminoacidi che creeranno proteine.
	 \end{enumerate}
	
\newpage
\section{Codice}
	Il codice è in python ed esegue:
	\begin{enumerate}
		\item pulisce la cartella di output per far spazio ai nuovi file, usando la libreria os.
		\item per ogni coppia di file in input contenendo l'analisi di allineamento eseguita usando i due tool:
		\begin{enumerate}
			\item salva il nome del file json in ooutput seguendo l'analisi.
			\item ne compara le differenze.
		\end{enumerate}
	\end{enumerate}
	\subsection{librerie}
		\begin{itemize}
			\item\textbf{re:} usato in \textbf{parsers.py} per dividere ogni linea tra: [id\_sequenza, sequenza, posizioni] nelle funzioni di parsing.
			\item\textbf{hashlib:} usato in \textbf{utils.py} per creare l'hash da inserire come nome ai file JSON di output.
			\item\textbf{json:} usato in \textbf{utils.py} per elaborare ulteriormente l'uotput al fine di comparare i 2 tools di allineamento.
			\item\textbf{datetime:} usato in \textbf{utils.py} per prendere il tempo da inserire nell'hash, al fine di rendere unico l'output ed evitare sovrascritture.
			\item\textbf{os:} usato in \textbf{utils.py, main.py} per gestire input e output. Anche per pulire la cartella input nel main.
		\end{itemize}
	\subsection{descrizione metodi}
		\textbf{runClustal}(inputFile, reference\_id, nseq = 3): funzione che esegue il parsing del file di allineamento clustal (inputFile), esegue il parsing degli allineamenti e li salva nel file di output. \newline nseq serve al parser in quanto la classe \textbf{ClustalParser} richiede il numero di sequenze da elaborare nelle sue funzioni.\newline
		\textbf{runMuscle}(inputFile, reference\_id, nseq = 3): funzione che esegue il parsing del file di allineamento muscle (inputFile), esegue il parsing degli allineamenti e li salva nel file di output. \newline nseq serve al parser in quanto la classe \textbf{ClustalParser} richiede il numero di sequenze da elaborare nelle sue funzioni.\newline
		\textbf{parse}(self, filename, reference=None, list=[]):  legge il file in ingresso e restituisce il file in input suddiviso in: reference, sequenze, lunghezza sequenze.\newline
		\textbf{save}(alignment, analyzer, reference\_id=None, tool=None, path=None): crea file json di output chiamato reference\_id\_hash-sha1 del tempo di inzio esecuzione concatenato con sequences\_ids delle sequenze.
		.\newline
		
\section{Output}	
	Come output vengono prodotti dei file json corrispondenti al nome reference\_seq\_hash.hexdigest(): l'id della reference accostata all'hash sha1 con encoding utf8 dell'id delle sequenze usate concatenate al tempo di inizio lavorazione (aggiunto per evitare di sovrascrivere l'output dello stessa serie di sequenze sia con muscle che con clustal).\newline
	l'oggetto JSON è così composto: \{ "reference": reference\_Id, \newline "analyzed\_sequences": insieme di sequenze analizzate, \newline
 	''unmatches": altri oggetti JSON = \newline\{ id (hash inizio mismatch + fine mismatch): "from": inizio mismatch, \newline
   "to": fine mismatch, \newline
   "sequences": lista sequenze in diverse tra loro in quell'intervallo \} \newline
 \}
\section{Conclusioni}

\end{document}