\documentclass[11pt,italian]{article}
\usepackage[T1]{fontenc}
\usepackage[utf8]{inputenc} %utf8 % lettere accentate da tastiera
\usepackage[italian]{babel} % lingua del documento
\usepackage{blindtext}
\usepackage{enumitem}
\usepackage{float}
\usepackage{upquote}
\usepackage{xcolor}   % for \textcolor
\usepackage[font=small,labelfont=bf,skip=10pt]{caption}
\usepackage{subcaption}
\setlength{\belowcaptionskip}{5pt}
\usepackage{listings}
\lstset{
  basicstyle=\small\ttfamily,
  otherkeywords={self},             % Add keywords here
  basicstyle=\small\ttfamily,
  columns=fullflexible,
  frame=single,
  breaklines=true,
  postbreak=\mbox{\textcolor{red}{$\hookrightarrow$}\space},
  tabsize=4, % tab space width
  showstringspaces=false, % don't mark spaces in strings
  numbers=left, % display line numbers on the left
  commentstyle=\color[HTML]{a0a1a7}, % comment color
  keywordstyle=\color[HTML]{40a3f5}, % keyword color
  stringstyle=\color{red}, % string color,
  emphstyle={\color[HTML]{40a3f5}}
}
\usepackage{hyperref}
\usepackage{cleveref}
\usepackage{graphicx}
\graphicspath{ {./images/} }

% Use lstinline as item in description
\makeatletter
\newcommand*{\lstitem}[1][]{%
  \setbox0\hbox\bgroup
    \patchcmd{\lst@InlineM}{\@empty}{\@empty\egroup\item[\usebox0]\leavevmode\ignorespaces}{}{}%
    \lstinline[#1]%
}
\makeatother

\title{Multiple Sequence Alignment (MSA) \\ di sequenze SARS-CoV-2}

\date{A.A.: 2019/2020}

\author{
    \textsc{Edoardo Silva} 816560 \\
    \textsc{Davide Marchetti} 815990
}

\begin{document}
\maketitle

\section{Abstract}
Date le variazioni sulle sequenze rilevate e catalogate nelle parti precedenti, costruiremo appositamente una matrice binaria di caratteri, evitando la generazione della "matrice proibita", da utilizzare per la realizzazione della filogenesi perfetta delle sequenze.

A partire dai file di output della prima parte del progetto, l'algoritmo genera una matrice binaria utilizzando come indici di riga gli identificativi delle sequenze e come colonne un identificativo univoco assegnato ad ogni variazione per poi estrarne la più grande matrice adatta alla generazione di un albero di una filogenesi perfetta.

Successivamente realizzeremo l'albero filogenetico con l'ausilio delle librerie \textbf{Bio} e \textbf{anytree}, confronteremo dell'albero ottenuto rispetto a quello prodotto con i tool di allineamento e analizzeremo i risultati ottenuti in relazione alle ipotesi avanzate dal report preliminare della prima parte di progetto per capire come il virus si sia introdotto nelle aree geografiche esaminate.

\newpage

\section{Albero filogenetico}
Un albero filogenetico è un albero che descrive la sequenze di eventi di speciazione che hanno portato alle specie attuali, dove con il termine speciazione si intende la differenziazione in due gruppi dove la differenza genica è predominante.

Il \lstinline{principio di Parsimonia} sancisce come ogni specie sia specificata da un insieme di caratteri o attributi.
A livello genomico, un carattere o attributo può assumere diversi significati:
\begin{itemize}
  \item ordine o composizione dei geni in un genoma
  \item composizione delle proteine
  \item cambiamenti genomici rari (RGC)
  \item acquisizione o perdita di introni
\end{itemize}

\noindent
La filogenesi perfetta deriva dall'unione di due principi:
\begin{itemize}
	\item \lstinline{Dollo parsimony}: un carattere può essere acquisito una sola volta ($0 \to 1$), ma può venire perso più volte ($1 \to 0$).
	\item \lstinline{Camin-Sokal parsimony}: un carattere può essere acquisito molteplici volte ($0 \to 1$), ma non può essere perso.
\end{itemize}
In una filogenesi perfetta risulta possibile quindi acquisire e perdere un carattere esclusivamente una volta.
Da questa affermazione ne deriva che un albero rappresentante una filogenesi perfetta avrà alcune caratteristiche:
\begin{enumerate}
	\item Ogni carattere identifica esattamente una ramificazione.
	\item L'albero ha il numero di foglie equivalente al numero di campioni analizzati, con ogni campione come foglia.
	\item Seguendo il percorso da una foglia fino alla radice si identificano i caratteri presenti nella sequenza posta sulla foglia di partenza.
\end{enumerate}

\newpage
\section{Algoritmo}
\subsection{Matrice delle variazioni}
Per la costruzione dell'albero abbiamo scelto di utilizzare ciascuna alterazione come identificativa di un singolo carattere, associandovi un codice univoco. La matrice così ottenuta pone sulle righe l'identificativo di ciascuna sequenza analizzata (indice $i$) e sulle colonne, denominate caratteri, le singole variazioni (indice $j$).

Ciascuna cella della matrice viene popolata con il valore \lstinline{1} qualora il $j$-esimo carattere (variazione) si manifesti nell'$i$-esima sequenza, altrimenti viene inserito il valore zero.

Procedendo al riempimento della matrice in questo modo si ottiene la matrice di caratteri riportata in \cref{fig:matrix-characters}.

\begin{figure}[H]
  \makebox[\textwidth][c]{\includegraphics[width=1.5\linewidth]{character_table.png}}
  \caption{Matrice di caratteri iniziale}
  \label{fig:matrix-characters}
\end{figure}

\subsection{Filogenesi perfetta}
Tuttavia, prima di procedere con la creazione dell'albero è necessario verificare che la matrice binaria generata in precedenza sia valida per una filogenesi perfetta, cioè che non sia contenuta una sottomatrice che sia la \lstinline{"matrice proibita"}.

Applicando l'algoritmo nel listato \ref{code:get_perfect_phylogeny_function} alla matrice binaria completa di tutti i caratteri, si ottiene la matrice illustrata in \cref{fig:matrix-perfect-phylo}.
Durante il procedimento vengono esclusi i caratteri: \lstinline{C3, C7, C12, C31} e \lstinline{C44}.
\begin{figure}[]
  \makebox[\textwidth][c]{\includegraphics[width=1.5\linewidth]{perfect_phylogeny_table.png}}
  \caption{Matrice di caratteri per filogenesi perfetta}
  \label{fig:matrix-perfect-phylo}
\end{figure}

\subsection{Generazione dell'albero}
A questo punto è possibile procedere con la ricostruzione dell'albero filogenetico a partire dalla \lstinline{candidate_matrix} e utilizzando funzioni definite nel file \lstinline{phylogeny.py} per costruire e visualizzare in output l'albero della filogenesi.

\subsubsection{Creazione dell'albero}
\noindent
La generazione dell'albero viene svolta come segue:
\begin{enumerate}
  \item Ordinamento decrescente delle colonne della matrice in base al numero di valori pari a \lstinline{1} presenti.
  \item A partire dal nodo \lstinline{root}, per ogni sequenza genera una sequenza di nodi uno per ogni variazione presente, collegandola alla variazione precedente o al nodo \lstinline{root}.

  Durante l'elaborazione delle sequenze successive, se esiste già un figlio del \lstinline{current_node} rappresentante la variazione attuale, questo viene riutilizzato, altrimenti viene creato un nuovo nodo e collegato come figlio di \lstinline{current_node}.

  \item Conversione dell'albero precedentemente generato in un \lstinline{newick_tree} elaborabile dalla libreria \lstinline{biopython.Phylo} usata per la visualizzazione.

  \item Inserimento delle informazioni aggiuntive sulle sequenze nelle foglie dell'albero. Questo step non può essere effettuato durante la conversione a \lstinline{newick_tree} perché alcuni caratteri usati verrebbero interpretati come parte della struttura dell'albero, alterando il risultato finale.

  \item Salvataggio e visualizzazione dell'albero filogenetico finale.
\end{enumerate}

\newpage
\section{Analisi dei risultati e conclusioni}
Oltre alla generazione dell'albero tramite il nostro script, sono stati prodotti due diversi alberi filogenetici a partire dagli allineamenti: in \cref{fig:tool-tree} é illustrato l'albero generato con i parametri di default, mentre in \cref{fig:tool-tree-skip} quello ottenuto abilitando il parametro \lstinline{Exclude Gaps=On}.

È necessario sottolineare che si è scelto di riportare esclusivamente gli alberi generati dall'allineamento effettuato con ClustalW poiché identici a quelli prodotti da MUSCLE.

\begin{figure}[H]
  \makebox[\textwidth][c]{\includegraphics[width=1\linewidth]{ncbi_tree_with_legend.png}}
  \caption{Albero generato con parametri di default}
  \label{fig:tool-tree}
\end{figure}
\noindent
La differenza così evidente di posizionamento della reference tra l'albero in \cref{fig:tool-tree} e quelli in \cref{fig:tool-tree-skip} avviene in quanto il parametro \lstinline{Exclude Gaps} ignora le colonne per le quali in almeno una delle sequenze si verifica una cancellazione (gap), forzando l'utilizzo esclusivamente delle colonne dove si ha una base in tutte le sequenze.

Per la generazione del nostro albero invece sono state utilizzate proprio le alterazioni che si posizionavano all'interno di geni.
Fin dalla prima fase del progetto abbiamo osservato come le cancellazioni avvengano agli estremi delle sequenze. Successivamente nella seconda parte del progetto abbiamo avuto conferma di quanto ipotizzato: nella quasi totalità dei casi le variazioni che coinvolgono geni sono sostituzioni.

Questo fornisce una possibile spiegazione sul perchè l'esclusione dei gap determini uno spostamento così marcato di alcune sequenze.

\begin{figure}[]
  \makebox[\textwidth][c]{\includegraphics[width=1\linewidth]{ncbi_tree_skip_gaps_with_legend.png}}
  \caption{Albero generato con \lstinline{Exclude Gaps=On}}
  \label{fig:tool-tree-skip}
\end{figure}

\noindent
Confrontando ora gli alberi generati dall'allineamento con ClustalW (\cref{fig:tool-tree,fig:tool-tree-skip}) e quello generato dal nostro script (\cref{fig:script-tree}), si riesce ad identificare un gruppo di sequenze in comune che copre le sequenze tra \lstinline{MT281530.2} e \lstinline{EPI_ISL_437512}.

\begin{figure}[]
  \makebox[\textwidth][c]{\includegraphics[width=1\linewidth]{perfect-phylogeny_with_legend.png}}
  \caption{Albero di output del nostro script}
  \label{fig:script-tree}
\end{figure}

\vspace{2mm}
\noindent
Osservando in dettaglio i tre alberi si possono notare alcuni dettagli interessanti:
\begin{itemize}
  \item Il nodo israeliano etichettato con \lstinline{MT276597.1}, includendo i gap presenta mutazioni differenti rispetto alle altre due sequenze israeliane. Questo potrebbe far presupporre che derivi da un ceppo diverso del virus o sia entrato nel paese da un'area geografica differente.
  \item La sequenza pakistana \lstinline{MT262993.1} presenta alterazioni differenti rispetto alle altre sequenze dello stesso paese. Come già osservato nell'analisi iniziale questa sequenza presenta esclusivamente cancellazioni e nessuna sostituzione.
  \item I rami restanti degli alberi nelle \cref{fig:tool-tree-skip,fig:script-tree} illustrano come nella stessa area geografica i genomi siano simili tra loro e condividano molte variazioni il che ci porta a ipotizzare che la diffusione all'interno del paese si sia sviluppata a partire da un singolo paziente per nazione e non da pazienti contagiati da mutazioni diverse del virus.
  \item I sequenziamenti rilevati in \lstinline{Israele} sembrano avere una possibile origine differente e quasi non condividere mutazioni simili con le sequenze degli altri paesi analizzati.
  \item I ceppi iraniani e turchi presentano mutazioni comuni e le sequenze di entrambi i paesi pare condividano una variazione con alcune sequenze rilevate in Pakistan.
\end{itemize}

\vspace{2mm}
\noindent
In conclusione, gli alberi nelle \cref{fig:script-tree,fig:tool-tree-skip,fig:tool-tree} risultano abbastanza simili, tuttavia presentano alcune differenze nel posizionamento di alcune sequenze.

Ipotizziamo che queste derivino principalmente dal fatto che l'albero in \cref{fig:script-tree} è costruito basandosi sul numero di alterazioni, mentre gli alberi in \cref{fig:tool-tree,fig:tool-tree-skip} vengono generati rispetto alle intere sequenze.

\newpage 
\subsection{Divisone del lavoro}
Durante la realizzazione del progetto entrambi i componenti del gruppo hanno partecipato attivamente alla sua realizzazione. In particolare:
\begin{itemize}
  \item \textbf{Edoardo Silva} si è occupato principalmente di recuperare i file esterni da elaborare e confrontare.
  \item \textbf{Davide Marchetti} si è occupato principalmente di generare i file di output.
  \item Entrambi hanno lavorato alla creazione ed elaborazione della matrice e dell'albero, con tutte le funzioni ausiliarie allo scopo del progetto.
\end{itemize}

\newpage
\section{Listati di codice}
\begin{lstlisting}[language=Python,caption=Caricamento dei file necessari per l'elaborazione,label=code:read_input_files]
reference_id = load_fasta_id(
  os.path.join('..', '..', 'project-1', 'input', 'reference.fasta')
)
sequence_ids = read_sequence_ids(paths=[
  os.path.join('..', '..', 'project-1', 'input', 'GISAID'),
  os.path.join('..', '..', 'project-1', 'input', 'ncbi'),
])
sequence_ids.insert(0, reference_id) #insert reference no variations

clustal_output = load_output('Clustal-NC_045512.2.json')
variations = clustal_output['unmatches'].items()
\end{lstlisting}

\begin{lstlisting}[caption=Generazione della matrice binaria di caratteri,label=code:creation table.csv,language=Python]
for key, value in variations:
  row = np.zeros(len(sequence_ids))
  indexes.append('C{}'.format(counter))
  for sequence in value['sequences']:
    row[sequence_ids.index(sequence)] = 1
  rows.append(row)
  counter += 1

trait_matrix = pd.DataFrame(rows, index=indexes, columns=sequence_ids, dtype=np.uint8).transpose()
trait_matrix = phylogeny.reorder_columns(trait_matrix, axis=0)
trait_matrix.to_csv(os.path.join('..', 'output', 'character_table.csv'))
\end{lstlisting}

\begin{lstlisting}[caption=Funzione di generazione della matrice binaria per filogenesi perfetta,label=code:get_perfect_phylogeny_function,language=Python]
def get_perfect_phylogeny_character_matrix(df):
  columns = df.columns
  candidate_matrix = df[columns[0:1]]
  for i in range(1, len(columns)):
    candidate_matrix = candidate_matrix.join(df[columns[i:i+1]])

    if phylogeny.is_forbidden_matrix(candidate_matrix):
      candidate_matrix = candidate_matrix.drop(labels=candidate_matrix.columns[-1], axis=1)

  return candidate_matrix
\end{lstlisting}

\begin{lstlisting}[caption=Funzione di creazione dell'albero,label=code:tree_creation_function,language=Python]
root = Node('root', edges={})
for i, row in df.iterrows():
  current_node = root

  for j in range(len(row)):
    # If alteration is present in the current sequence
    if row.iloc[j]:
      # If current_node has a link to the variation with label=j
      if j in current_node.edges:
        # Follow the same path without creating new nodes
        current_node = current_node.edges[j]
      else:
        u = Node('U-{}'.format(row.index[j]), edges={})
        current_node.parent = current_node
        current_node.edges[j] = u
        current_node = u

  Node(i, parent=current_node)
\end{lstlisting}

\begin{lstlisting}[caption=Funzione di conversione in Newick Tree,label=code:newick_tree_function,language=Python]
def to_newick_tree(node):
  if node.is_leaf:
    return node.name
  return '({})'.format(','.join([ to_newick_tree(child) for child in node.children ]))

newick_string = to_newick_tree(root)
tree = Phylo.read(io.StringIO(newick_string), 'newick')
\end{lstlisting}

\begin{lstlisting}[caption=Funzione per l'inserimento dei dati nelle foglie dell'albero,label=code:merge_sequence_data_function,language=Python]
def merge_sequences_data(node, sequences_data):
  if node.is_terminal():
    data = sequences_data[node.name]
    node.name = '{}\n {} in {}'.format(node.name, data['date'], data['location'])
    return

  [ merge_sequences_data(child, sequences_data) for child in node.clades ]

root = newick_tree.clade
merge_sequences_data(root, sequences_data)
\end{lstlisting}

\end{document}