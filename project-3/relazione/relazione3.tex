\documentclass[11pt,italian]{article}
\usepackage[T1]{fontenc}
\usepackage[utf8]{inputenc} %utf8 % lettere accentate da tastiera
\usepackage[italian]{babel} % lingua del documento
\usepackage{blindtext}
\usepackage{enumitem}
\usepackage{float}
\usepackage{xcolor}   % for \textcolor
\usepackage[font=small,labelfont=bf,skip=10pt]{caption}
\usepackage{subcaption}
\setlength{\belowcaptionskip}{5pt}
\usepackage{listings}
\lstset{
  basicstyle=\ttfamily,
  columns=fullflexible,
  frame=single,
  breaklines=true,
  postbreak=\mbox{\textcolor{red}{$\hookrightarrow$}\space},
}
\usepackage{hyperref}
\usepackage{cleveref}
\usepackage{graphicx}
\graphicspath{ {./images/} }

% Use lstinline as item in description
\makeatletter
\newcommand*{\lstitem}[1][]{%
  \setbox0\hbox\bgroup
    \patchcmd{\lst@InlineM}{\@empty}{\@empty\egroup\item[\usebox0]\leavevmode\ignorespaces}{}{}%
    \lstinline[#1]%
}
\makeatother

\title{Multiple Sequence Alignment (MSA) \\ di sequenze SARS-CoV-2}

\date{A.A.: 2019/2020}

\author{
    \textsc{Edoardo Silva} 816560 \\
    \textsc{Davide Marchetti} 815990
}

\begin{document}
\maketitle

\section{Abstract}
Il lavoro svolto è diviso in 2 parti e il codice in 4 sezioni:
\subsection{Matrice delle variazioni}
Qui si trovano le prime 2 parti del codice: il recupero per intero delle 14 sequenze usate e la lettura dei file di variazione generati in output nella parte1
\begin{lstlisting}[basicstyle=\small\ttfamily,caption=Porzione di ciclo,label=code:Read fasta = [], language = Python]

    reference_id = load_fasta_id(os.path.join('..', '..', 'project-1', 'input', 'reference.fasta'))
    sequence_ids = read_sequence_ids(paths=[
        os.path.join('..', '..', 'project-1', 'input', 'GISAID'),
        os.path.join('..', '..', 'project-1', 'input', 'ncbi'),
    ])
    sequence_ids.insert(0, reference_id)    #insert reference no variations
    
    ...

    clustal_output = load_output('Clustal-NC_045512.2_2020-05-30_16-51.json')
    variations = clustal_output['unmatches'].items()

\end{lstlisting}
\newpage
Infine nella parte 3 del nostro codice generiamo la tabella delle variazioni \textbf{`table.csv'} per variazione, per poi trasporla.

\begin{lstlisting}[basicstyle=\small\ttfamily,caption=Porzione di ciclo,label=code:creation table.csv = [], language = Python]

...

for key, value in variations:
        row = np.zeros(len(sequence_ids))
        indexes.append('C{}'.format(counter))
        for sequence in value['sequences']:
            row[sequence_ids.index(sequence)] = 1
        rows.append(row)
        counter += 1
        
...

trait_matrix = pd.DataFrame(rows, index=indexes, columns=sequence_ids, dtype=bool).transpose()
trait_matrix = phylogeny.reorder_columns(trait_matrix, axis=0, ascending=False)
trait_matrix.to_csv(os.path.join('..', 'output', 'table.csv'))

\end{lstlisting}

%%insert image 'table.csv'%%

\subsection{Albero filogenetico}
La quarta e ultima parte del codice consiste nel ricostrure l'albero filogenetico prendendo l'output del punto 3 e facendo estensivo uso di funzioni create per questo lavoro nel file \textbf{`phylogeny.py'} per:
\begin{enumerate}
	\item generare matrice compatibile con filogenesi perfetta.
	\item assicurarsi che tutto abbia funzionato e non sia più presente matrice proibita.
	\item generare e stampare in output l'albero delle sequenze.
\end{enumerate}
\begin{lstlisting}[basicstyle=\small\ttfamily,caption=Porzione di ciclo,label=code:creation table.csv = [], language = Python]
candidate_matrix = get_perfect_phylogeny_character_matrix(trait_matrix)
if phylogeny.is_forbidden_matrix(candidate_matrix):
        raise Exception('Invalid perfect phylogeny matrix')
phylogeny.build_tree(candidate_matrix)
\end{lstlisting}

\section{conclusione}


\end{document}